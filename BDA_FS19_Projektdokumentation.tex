\documentclass[
	a4paper
]{scrreprt}

%%% PACKAGES %%%

% PDF/A Compliance
\usepackage[a-2b,latxmp]{pdfx}

% add unicode support and use german as language
\usepackage[utf8]{inputenc}
\usepackage[ngerman]{babel}

% Use Helvetica as font
\usepackage[scaled]{helvet}
\renewcommand\familydefault{\sfdefault}
\usepackage[T1]{fontenc}

% Better tables
\usepackage{tabularx}

% Better enumerisation env
\usepackage{enumitem}

% Use graphics
\usepackage{graphicx}

% Have subfigures and captions
\usepackage{subcaption}

% Be able to include PDFs in the file
\usepackage{pdfpages}

% Have custom abstract heading
\usepackage{abstract}

% Need a list of equation
\usepackage{tocloft}
\usepackage{ragged2e}

% Better equation environment
\usepackage{amsmath}

% Symbols for most SI units
\usepackage{siunitx}

\usepackage{csquotes}

% Clickable Links to Websites and chapters
\usepackage{hyperref}

% Change page rotation
\usepackage{pdflscape}

% Symbols like checkmark
\usepackage{amssymb}
\usepackage{pifont}

\usepackage[absolute]{textpos}

% Glossary, hyperref, babel, polyglossia, inputenc, fontenc must be loaded before this package if they are used
\usepackage{glossaries}
% Redefine the quote charachter as we are using ngerman
\GlsSetQuote{+}
% Define the usage of an acronym, Abbreviation (Abbr.), next usage: The Abbr. of ...
\setacronymstyle{long-short}

% Bibliography & citing
\usepackage[
	backend=biber,
	style=apa,
	bibstyle=apa,
	citestyle=apa,
	sortlocale=de_DE
	]{biblatex}
\addbibresource{Referenzen.bib}
\DeclareLanguageMapping{ngerman}{ngerman-apa}

%%% COMMAND REBINDINGS %%%
\newcommand{\tabitem}{~~\llap{\textbullet}~~}
\newcommand{\xmark}{\ding{55}}
\newcommand{\notmark}{\textbf{\textasciitilde}}
% Pro/Con item https://tex.stackexchange.com/questions/145198/change-the-bullet-of-each-item#145203
\newcommand\pro{\item[$+$]}
\newcommand\con{\item[$-$]}

% Define list of equations - Thanks to Charles Clayton: https://tex.stackexchange.com/a/354096
\newcommand{\listequationsname}{\huge{Formelverzeichnis}}
\newlistof{myequations}{equ}{\listequationsname}
\newcommand{\myequations}[1]{
	\addcontentsline{equ}{myequations}{\protect\numberline{\theequation}#1}
}
\setlength{\cftmyequationsnumwidth}{2.3em}
\setlength{\cftmyequationsindent}{1.5em}

% Usage {equation}{caption}{label}
% \indexequation{b = \frac{\pi}{\SI{180}{\degree}}\cdot\beta\cdot 6378.137}{Bogenlänge $b$ des Winkels $\beta$ mit Radius 6378.137m (Distanz zum Erdmittelpunkt am Äquator)}{Bogenlaenge}
\newcommand{\indexequation}[3]{
	\begin{align} \label{#3} \ensuremath{\boxed{#1}} \end{align}
	\myequations{#3} \centering \small \textit{#2} \normalsize \justify }

% Todolist - credit to https://tex.stackexchange.com/questions/247681/how-to-create-checkbox-todo-list
\newlist{todolist}{itemize}{1}
\setlist[todolist]{label=$\square$}

% Nested Enumeratelist credit to https://tex.stackexchange.com/a/54676
\newlist{legal}{enumerate}{10}
\setlist[legal]{label*=\arabic*.}

%%% PATH DEFINITIONS %%%
% Define the path were images are found
\graphicspath{{./img/}{./appendix/}}

%%% GLOSSARY ENTRIES %%%
\makeglossaries
\newacronym{RFID}{RFID}{Radio-Frequency Identification}
\newglossaryentry{HF}{name={HF},description={High Frequency, RFID Tags im Frequenzbereich von 3-30MHz}}
\newglossaryentry{UHF}{name={UHF},description={Ultra High Frequency, RFID Tags im Frequenzbereich von 0.3-30GHz}}
\newglossaryentry{DLL}{name={DLL},description={Dynamic Link Library, eine dynamisch ladbare Programmbibliothek}}

%%% DOCUMENT %%%

\begin{document}

%%% Fallback DocumentVersion if Builded local
\providecommand{\docversion}{0.0-localBuild}
\begin{titlepage}
	\begin{textblock*}{5cm}[0,0](15.1cm,1cm)
		\includegraphics[keepaspectratio,width=5cm]{img/HSLU_Logo}
	\end{textblock*}
	\begin{center}
		\vspace*{5cm}
		\Huge{\textbf{Suche von mit RFID ausgerüsteten Einzelexemplaren im vollautomatischen Behälter-Hochregallager}} \\
		\vspace{0.5em}
		\Large{Bachelordiplomarbeit FS2019}\\
		\vspace{3em}
		\LARGE{Pascal Baumann, Dane Wicki}\\
		\vspace{1em}
		\Large{Betreuer: Martin Jud}\\
		\vfill
		\large{Hochschule Luzern - Departement Informatik}\\
		\large{\today}\\
		\large{Version \docversion}
	\end{center}
	\begin{textblock*}{5cm}[0,0](15.3cm,277mm)
		\includegraphics[keepaspectratio,width=5cm]{img/FHZ_Logo}
	\end{textblock*}
\end{titlepage}

\newpage

\pagenumbering{gobble}

\begin{textblock*}{5cm}[0,0](15cm,0.7cm)
	\includegraphics[keepaspectratio,width=2.7cm]{img/HSLU_Logo_Header}
\end{textblock*}

\vspace*{1.35cm}

\noindent
\textbf{\Large{Bachelorarbeit an der Hochschule Luzern - Informatik}}

\vspace{0.6cm}
\noindent
\textbf{Titel:} RFID markierte Exemplare

\vspace{0.6cm}
\noindent
\textbf{Student 1:} Pascal Baumann

\vspace{0.6cm}
\noindent
\textbf{Student 2:} Dane Wicki

\vspace{1cm}
\noindent
\textbf{Studiengang:} BSc Informatik

\vspace{0.6cm}
\noindent
\textbf{Abschlussjahr:} 2019

\vspace{0.6cm}
\noindent
\textbf{Betreuungsperson:} Martin Jud

\vspace{0.6cm}
\noindent
\textbf{Experte:} Urs Gehrig

\vspace{0.6cm}
\noindent
\textbf{Codierung / Klassifizierung der Arbeit:}

\begin{todolist}
	\item \textbf{A: Einsicht (Normalfall)}
	\item \textbf{B: Rücksprache}\hspace*{0.7cm}(Dauer:\hspace*{1cm} Jahr / Jahre)
	\item \textbf{C: Sperre}\hspace*{1.865cm}(Dauer:\hspace*{1cm} Jahr / Jahre)
\end{todolist}

\vfill

\noindent
\textbf{Eidesstattliche Erklärung}
\\
Ich erkläre hiermit, dass ich/wir die vorliegende Arbeit selbständig und ohne unerlaubte fremde Hilfe angefertigt haben, alle verwendeten Quellen, Literatur und andere Hilfsmittel angegeben haben, wörtlich oder inhaltlich entnommene Stellen als solche kenntlich gemacht haben, das Vertraulichkeitsinteresse des Auftraggebers wahren und die Urheberrechtsbestimmungen der Fachhochschule Zentralschweiz (siehe Markblatt «Studentische Arbeiten» auf MyCampus) respektieren werden.

\vspace{1em}

\noindent
\begin{tabularx}{\textwidth}{@{}lX}
	&\\
	Ort / Datum, Unterschrift: &  \\
	\cline{2-2}
	&\\[0.5cm]
	Ort / Datum, Unterschrift: &  \\
	\cline{2-2}
\end{tabularx}

\begin{textblock*}{5cm}[0,0](14.93cm,277mm)
	\includegraphics[keepaspectratio,width=5cm]{img/FHZ_Logo}
\end{textblock*}

\newpage

\begin{textblock*}{5cm}[0,0](15cm,0.7cm)
	\includegraphics[keepaspectratio,width=2.7cm]{img/HSLU_Logo_Header}
\end{textblock*}

\noindent
\textbf{Abgabe der Arbeit auf der Portfolio Datenbank}

\vspace{0.5em}

\noindent
\textbf{Bestätigungsvisum Studentin / Student}
\\
\noindent
Ich bestätige, dass ich die Bachelorarbeit korrekt gemäss Merkblatt auf der Portfolio Datenbank abgelegt habe. Die Verantwortlichkeit sowie die Berechtigungen habe ich abgegeben, so dass ich keine Änderungen mehr vornehmen kann oder weitere Dateien hochladen kann.

\vspace{0.7em}

\noindent
\begin{tabularx}{\textwidth}{@{}lX}
	&\\
	Ort / Datum, Unterschrift: &  \\
	\cline{2-2}
	&\\[0.5cm]
	Ort / Datum, Unterschrift: &  \\
	\cline{2-2}
\end{tabularx}

\vspace{0.8cm}
\noindent
\textbf{Verdankung}
\\
An dieser Stelle möchten wir uns bei all denjenigen bedanken, die uns bei der Erstellung dieser Bachelorarbeit unterstützt haben. Als Erstes möchten wir uns bei unserem Betreuer Martin Jud bedanken, der unsere Bachelorarbeit unterstützt und begutachtet hat. Für die hilfreichen Empfehlungen zur Struktur dieser Arbeit und den Hilfestellungen zur Anschaffung der Versuchshardware möchten wir uns herzlich bedanken. Weiter möchten wir uns bei Urs Gehrig für seinen Einsatz als Experten für unsere Arbeit bedanken. Einen Dank auch an Herrn Mike Märki, welcher diese Arbeit überhaupt erst ermöglicht hat, und uns erlaubte unsere Referenzimplementation vor Ort zu testen. Ganz herzlich möchten wir uns auch bei Jeremy Meier bedanken, welcher uns in der Architektur beraten und ein Review der Systemspezifikation geliefert hat.

\vspace{0.8cm}
\noindent
\textbf{Eingangsvisum (durch das Sekretariat auszufüllen):}

\noindent
\renewcommand{\arraystretch}{2}
\begin{tabularx}{\textwidth}{@{}lXlX}
	Rotkreuz, den & & Visum: & \\
	\cline{2-2}
	\cline{4-4}
\end{tabularx}
\renewcommand{\arraystretch}{1}

% Uncomment for print_version

% \vfill
% \noindent
% \textbf{Hinweis}: Die Bachelorarbeit wurde von keinem Dozierenden nachbearbeitet. Veröffentlichungen (auch auszugsweise) sind ohne das Einverständnis der Studiengangleitung der Hochschule Luzern – Informatik nicht erlaubt.

% \vspace{1em}
% 
% \noindent
% \textbf{Copyright} © 2019 Hochschule Luzern - Informatik
% 
% \vspace{1em}
% \noindent
% Alle Rechte vorbehalten. Kein Teil dieser Arbeit darf ohne die schriftliche Genehmigung der Studiengangleitung der Hochschule Luzern - Informatik in irgendeiner Form reproduziert oder in eine von Maschinen verwendete Sprache übertragen werden.

\begin{textblock*}{5cm}[0,0](14.93cm,277mm)
	\includegraphics[keepaspectratio,width=5cm]{img/FHZ_Logo}
\end{textblock*}


\pagenumbering{Roman}

\begin{abstract}
	Das Ziel dieser Arbeit ist es, mittels einer Machbarkeitsstudie zu ermitteln, ob es technisch und finanziell möglich ist, die vollautomatische Suche eines Einzelexemplars in der Kooperativen Speicherbibliothek in Büron zu realisieren. Über 16 Wochen wurden Recherchen durchgeführt, zwei Konzepte ausgearbeitet, über identifizierte Unklarheiten Versuche mit erworbener Hardware durchgeführt und aufgrund der Ergebnisse dieser eine Machbarkeitsstudie geschrieben. In dieser wird beschrieben, wie das Problem von deplatzierten Einzelexemplaren vor der Einlagerung gelöst werden kann. Dabei werden Aufwendungen von rund 20'000 Franken für die Entwicklung identifiziert. In dieser Projektdokumentation wird beschrieben, wie der Stand der Technik in Bezug auf RFID im Moment aussieht und die gewählte, agile Projektmethode erläutert. Es werden die zwei erarbeiteten Konzepte vorgestellt, von welches eines in der Machbarkeitsstudie berücksichtigt wurde, und die Realisierung einer Referenzimplementation dieses Konzeptes beschrieben. Diese Referenzimplementation wurde vor Ort getestet und damit das gewählte Konzept validiert. Weiter wird in der Evaluation und Validation dargestellt, dass die identifizierten Anforderungen erfolgreich erreicht werden konnten und was im Projektverlauf funktioniert hat. Dabei werden sowohl die technischen Hilfsmittel, wie auch die entwickelten Arbeitsprozesse betrachtet. Im Ausblick wird das weitere Vorgehen für das Folgeprojekt erläutert und Empfehlungen für das Folgeteam ausgesprochen. Die Arbeit schliesst mit dem Fazit des Projektteams ab.
\end{abstract}

\tableofcontents

\clearpage
\pagenumbering{arabic}

\chapter{Einleitung}

\section{Aufgabenstellung und Zielsetzung}


\chapter{Stand der Technik}
\label{ch:StandDerTechnik}

Das erste \gls{rfid}-System wurde von den Allierten im Zweiten Weltkrieg eingesetzt. Deren Freund-Feind-Erkennung funktionierte über eine passive Reflektion der Radarwellen, welche auf die Frequenz der Radarsender geeicht war. Verbündete Flieger ergaben dadurch ein viel stärkeres Signal und waren als hellere Punkte erkennbar (\cite{chawla2007}, \cite{uswardep1946_3}). In den 1960er wurden in den USA verschiedenste Patente angemeldet, welche sich das Prinzip der elektromagnetischen Induktion zunutze machen, um sich gegenüber einem Sender zu identifizieren. Die Adaption der Technologie liess jedoch auf sich warten, \citeauthor{want2004} identifiziert als Grund dafür die fehlende Marktviabilität, die Ausgereiftheit der Technologie selbst und die Adaptionskosten \parencite{want2004}. All dies hat sich in den 60 Jahren seither geändert und \gls{rfid} ist eine etablierte Technologie, welche in verschiedensten Sparten eingesetzt wird.

\section{Technologische Grundlagen}

% Funktionsweise Transponder / Interrogator

% Active/Passive

Bei der Funktionsweise von \gls{rfid} Tags muss man zwischen zwei Funktionalitäten unterscheiden, welche auf die Entfernung zwischen Transponder und Interrogator abhängig sind. Im Nahbereich (engl. Near-Field \gls{rfid}) funktioniert die Stromversorgung über magnetische Induktion (die Arbeitsspannung der Chips liegt im Mikro- bis Miliwattbereich). Die Kommunikation zwischen Interrogator und Transponder wird über "load modulation"\ realisiert. Dies bedeutet, dass der Transponder, aktiviert durch das Feld des Interrogator, selber beginnt ein Feld auszustrahlen. Dadurch entstehen Interferenzen im Feld welche sich durch minimale Spannungsänderungen in der Spule des Interrogators messen lassen \parencite{want2006}. Die Distanz des Nahfelds ist durch die Gleichung \ref{NearFieldEM} gegeben. Für \gls{HF} Tags (wie diejenigen die auch in der Speicherbibliothek verwendet werden) ist die Betriebsfrequenz durch den ISO Standard 18000-3 auf 13.56MHz festgelegt und ergibt damit eine Distanz von 3.519m für das Nahfeld.

\indexequation{d=\frac{\lambda}{2\pi}}{Reichweite des reaktiven Nahfelds}{NearFieldDistance}
\indexequation{\lambda=\frac{c}{f}}{Definition der Wellenlänge $\lambda$ in Abhängigkeit der Lichtgeschwindigkeit c}{Wavelength}
\indexequation{d=\frac{c}{2\lambda\pi}}{Definition der Wellenlänge in Gleichung \ref{NearFieldDistance} eingesetzt}{NearFieldEM}

Im Fernfeldbereich erhält der \gls{rfid} Tag direkt über die ausgestrahlte Elektromagnetischen Wellen. Die Abnahme der Energiedichte auf Distanz ist dabei proportional zu $\frac{1}{r^2}$. Dennoch ist es durch Fortschritte in der Miniaturisierung und besserer Energieeffizienz moderner Halbleiter und Chips möglich dadurch \gls{UHF} Tags mit Strom zu versorgen. Die Kommunikation funktioniert mittels "back scattering"\ - eine Antenne welche auf eine bestimmte Frequenz eingestellt ist, absorbiert den Grossteil der Wellen die gesendet werden. Passt jedoch die Impedanz nicht genau, so reflektiert die Antenne ein Teil des Signals an die Quelle, den Interrogator, zurück. Durch das Anpassen der Impedanz der Antenne über die Zeit, kann mehr oder weniger des Signals reflektiert und so eine Nachricht codiert werden \parencite{want2006}.

\begin{figure}[htb]
	\centering
	\begin{subfigure}[b]{0.8\linewidth}
		\centering
		\includegraphics[keepaspectratio,width=\linewidth]{InductionCoupling}
		\caption{Magnetische Induktion und "load modulation"}
	\end{subfigure}
	\begin{subfigure}[b]{0.8\linewidth}
		\centering
		\includegraphics[keepaspectratio,width=\linewidth]{Backscattering}
		\caption{EM Wellentransmission und "backscattering"}
	\end{subfigure}
	\caption{Funktionsweisen von Nah- und Fernfeld \gls{rfid} Tags graphisch dargestellt \parencite{want2006}}
\end{figure}

\section{Technische Konzepte}

% CSMA Strategies

% Charakteristiken der unterschiedlichen Frequenzen

\section{Anwendungen}

% NFC, Regulatorische Vorschriften v.a. im Bezug zur Schweiz

\chapter{Methode}

\section{Projektinformationen}

\subsection{Vorgehensmodell}

Alle Wirtschaftsprojekte an der Hochschule Luzern fallen in eine der folgenden Kategorien:

\begin{enumerate}
	\item Einsatz von Standardsoftware und Services
	\item Software- und Produktentwicklung
	\item Innovationsprojekt
	\item IT-Infrastrukturentwicklung
	\item Strukturierte Analyse und Konzeption von Systemen und Abläufen
\end{enumerate}

Dabei ist dieses Projekt als Innovationsprojekt und Softwareentwicklung klassifiziert worden. Wir erwarteten daher unter anderem, eine Evaluation, Recherchen und weitere Unbekannten. Um auf diese eingehen zu können, entschied sich das Team dafür die hybride, inkrementelle Agile Methode zu verwenden.

\subsection{Agile Projektmethode}

Die agile Projektmethode zielt darauf ab in einem ungewissen und sich verändernden Umfeld zu bestehen. Insbesondere bedeutet dies, das auf sich verändernde Voraussetzungen schnell reagiert werden kann und dabei ein funktionierendes Produkt entsteht \parencite{AgileAlliance2015}. Dies soll durch eine enge Zusammenarbeit mit dem Auftraggeber und guter teaminterner Kommunikation erreicht werden.

\parencite{BaumannWicki2018}

\subsection{Ermittlung offener Projektrahmenbedingungen}
\label{ch:evaluation}

\subsection{Projektanforderungen}
Mittels einer Machbarkeitsstudie und einem Proof of Concept soll untersucht werden ob es möglich ist bis zu 120 \gls{RFID} Tags in einem Behälter mit der Dimension 600x400x320mm zu identifizieren.

\begin{itemize}
	\item Es sollen mindestens zwei Lösungskonzepte für eine als Auswahl der Machbarkeitsstudie entwickelt werden.
	\item Die Lösungskonzepte müssen auf deren technische Realisierbarkeit untersucht werden.
	\item Es muss mindestens ein entwickeltes Konzept für die Machbarkeitsstudie verwendet werden.
	\item Die Machbarkeitsstudie muss eine Kostenrechnung für die Lösungsansätze beinhalten.
	\item Es soll eine MVP entwickelt werden, welches vom Kunde verwendet werden kann.
\end{itemize}

\subsubsection{Anforderungen an Lösungsansätze, Proof of Concept und MVP}

\begin{itemize}
	\item Die Lösungskonzepte müssen mit dem Lagersystem kommunizieren können
	\item Die Lösungskonzepte müssen die \gls{RFID} Tags in weniger als 1 Sekunden identifizieren können.
	\item Die Lösungskonzepte müssen für das bestehende Hochregallager der Speicherbibliothek verwendbar sein.
	
	\item Das Proof of Concept muss technisch aufzeigen, wie viele \gls{RFID} Tags in einer Sekunde gelesen werden können.
	\item Das Proof of Concept soll eine \gls{RFID} Lesezuverlässigkeit von 95\% aufweisen.
	
	\item Das MVP soll mit der Datenbank des Lagersystems kommunizieren können.
	\item Das MVP soll in einem von Störfaktoren bereinigten Zustand die gleiche Anzahl \gls{RFID} Tags lesen können wie im Proof of Concept definiert.
	\item Das MVP soll erkennen, wenn eine Box ein Exemplar enthält, welches nicht dieser Box zugehörig ist und dies als eine Unstimmigkeit markieren.
	\item Das MVP soll in der Lage sein, dem Endbenutzer in beliebiger Form mitzuteilen, welcher Behälter eine Unstimmigkeit enthält.
\end{itemize}

\subsection{Einschränkungen und Abgrenzungen}
In diesem Projekt wurde explizit auf eine Implementation mit dem Lagerverwaltungssystem verzichtet, da der Fokus auf der Machbarkeitsstudie lag. Es wurde diese Implementation zwar berücksichtigt, aber sie stellt eine Unbekannte für das Folgeprojekt dar. Weiter wurde explizit die Entwicklung des Prototypen zur Produktionsreife als nicht Teil des Projektes verstanden. Die Referenzimplementation wurde als Validierung der Konzepte und Versuche verstanden, und nicht als Prototyp zur Einbindung in die Umgebung.

Für dieses Projekt wurde auch die Einbindung in die Datenbank des Lagerverwaltungssystem durch eine Momentanaufnahme als Dump abstrahiert, eine dynamische Einbindung ist daher auch Teil des Folgeprojektes.

Einschränkungen ergaben sich vor allem durch das Material welche durch das beschränkte Budget erworben werden konnte. Durch die Beschaffung über einen alternativen Hersteller, ergab sich eine kleinere Reichweite als gewünscht. Die Resultate können daher auch nur explizit für diese spezifische Hardware gewährleistet werden.
Weiter bestand Teamintern ein kleiner Wissensstand über die Durchführung einer Machbarkeitsstudie zu Beginn des Projektes, welcher im Laufe des Projektes zuerst erarbeitet werden musste.

\section{Machbarkeitsstudie}
Die Machbarkeitsstudie diente dazu ein ausgewähltes Konzept zu validieren und auf deren Machbarkeit zu untersuchen. Insbesondere wurde auf die technische und wirtschaftliche Machbarkeit wert gelegt. Da die Erarbeitung einer Machbarkeitsstudie eine Neuheit für beide Teammitglieder war, wurde zuerst in einer Recherchenphase offene Fragen geklärt. Dabei wurde eine Anleitung zur Erarbeitung einer Machbarkeitsstudie erstellt (siehe Kapitel \ref{app:ch:AnleitungMachbarkeitsstudie}). Diese stützt sich in grossen Teilen auf eine Anleitung des US Departement of Agriculture \parencite{Matson2000}.


\chapter{Ideen und Konzepte}

\section{Grundidee}

\section{Lösungskonzept 1}

\section{Lösungskonzept 2}


\chapter{Realisierung}

\newpage
-\section{Systemspezifikation}
\label{sec:SysSpec}

\subsection{Anforderungen}
\label{ch:Anforderungen}

\subsection{Kontext}

\subsection{Komponentendesign}

\subsection{Architektur \& Design}

\subsection{Interne Schnittstellen}

\subsection{Klassendiagramm}

\subsection{Anforderungen der Software}

\subsection{Umsetzung Programmierung}

\subsection{Testing}


\chapter{Evaluation und Validation}
\label{ch:Eval}
In diesem Kapitel wird evaluiert, ob das Projekt und dessen produzierte Artefakte die Anforderungen erfüllen. Dazu wurden alle Anforderungen nochmals zusammengetragen und jede einzelne mit dem jeweiligen Artefakt überprüft, das Resultat wurde schliesslich in Form einer einfach lesbaren Tabelle festgehalten. Aus dem Vergleich der konnte schliesslich festgestellt werden, dass alle priorisierten Anforderungen erfüllt werden konnten. Daher kann dieses Projekt als erfolgreich betrachtet werden.

\section{Vergleich mit Anforderungen}
\label{sec:VergleichAnforderungen}
Hier werden die Anforderungen aufgelistet und ob diese erfüllt wurden oder nicht. Sofern diese nicht erfüllt wurden, wird für deren Erfüllung eine potenzielle Lösungsidee beschrieben. Für die Verifikation wurden die Testfälle überprüft.

Legende:
\begin{itemize}[label={}, noitemsep]
	\item <<\checkmark >> = Anforderung erfüllt
	\item <<\xmark>> = Anforderung nicht erfüllt
	\item <<\notmark>> Wurde aus den beschriebenen Gründen nicht Implementiert.
\end{itemize}


\begin{tabularx}{\textwidth}{l l X}
	\hline
	\checkmark & 1   & Es sollen mindestens zwei Lösungskonzepte für eine als Auswahl der Machbarkeitsstudie entwickelt werden. \\
	\hline
	\checkmark & 2   & Die Lösungskonzepte müssen auf deren technische Realisierbarkeit untersucht werden. \\
	\hline
	\checkmark & 3   & Es muss mindestens ein entwickeltes Konzept für die Machbarkeitsstudie verwendet werden. \\
	\hline
	\checkmark & 4   & Die Machbarkeitsstudie muss eine Kostenrechnung für die Lösungsansätze beinhalten. \\
	\hline
	\checkmark & 5   & Es soll eine Referenzimplementation entwickelt werden, welche vom Kunde verwendet werden kann. \\
	\hline
	\checkmark & 6.1 & Die Lösungskonzepte müssen mit dem Lagersystem kommunizieren können \\
	\hline
	\checkmark & 6.2 & Die Lösungskonzepte müssen die RFID Tags in weniger als 1 Sekunden identifizieren können. \\
	\hline
	\checkmark & 6.3 & Die Lösungskonzepte müssen für das bestehende Hochregallager der Speicherbibliothek verwendbar sein. \\
	\hline
	\checkmark & 7.1 & Das Proof of Concept muss technisch aufzeigen, wie viele RFID Tags in einer Sekunde identifiziert werden können. \\
	\hline
	\checkmark & 8.1 & Die Referenzimplementation ist in der Lage die Buch ID eines Exemplares über RFID auszulesen. \\
	\hline
	\checkmark & 8.2 & Die Referenzimplementation soll erkennen, wenn eine Box ein Exemplar (eines, welches mit RFID ausgestattet ist und technisch auch Lesbar ist) enthält, welches nicht dieser Box zugehörig ist. \\
	\hline
	\checkmark & 8.3 & Die Referenzimplementation soll jede erkannte Unstimmigkeit (Exemplar, welches nicht zu diesem Behälter gehört) in einem Logdokument persistieren. \\
	\hline
	\checkmark & 8.4 & Die Referenzimplementation soll in der Lage sein, dem Endbenutzer, in graphischer Form, durch eine Konsolen-Ausgabe, mitzuteilen, welcher Behälter eine Unstimmigkeit enthält. \\
	\hline
	? & 8.5 & Die Referenzimplementation soll die unter Laborbedingungen erhaltenen Resultate unter Realbedingungen verifizieren. \\
	\hline
	\notmark & 8.6 & Die Referenzimplementation soll mit einer Oracle Datenbank kommunizieren können. \\
	\hline
\end{tabularx}

\subsection{Begründung des Nichtumsetzens von 8.6}
Während der Implementationsphase konnte festgestellt werden, dass die Einbindung der Oracle Datenbank mit grösserem Zeitaufwand verbunden ist, als dem Team noch zur Verfügung stand. Der Umstand, dass diese Anforderung in der Meilensteinsitzung 3 mit einer geringen Priorität versehen wurde (siehe Anhang \ref{app:sec:protokollMeilenstein3}), führte dazu, dass diese nicht umgesetzt wurde.

\section{Technische Aspekte}
\subsection{ContinuousIntegration / ContinuousDeployment}
Im Projekt wurde durchgehend Travis für CI/CD verwendet, dies bedeutete, dass sowohl die Dokumentation wie auch die erstellten Applikationen bei jedem Push gebaut wurden, und auf dem Masterbranch direkt als Prerelease deployed wurde.

Es konnten so Fehler direkt erkannt und behoben werden, und es bestand die Möglichkeit auf alte Versionen direkt zuzugreifen. Wir beurteilen dieses Hilfsmittel als extrem effektiv.

\subsection{\gls{HF} \gls{RFID} Hardware}
Die evaluierte \gls{HF} \gls{RFID} Hardware der Hyintech erreichte nicht die gewünschte Reichweite und funktionierte nur auf einem 32bit Windows System (aufgrund der gelieferten DLL). Sie ermöglichte es dennoch, unsere Versuche durchzuführen und eine aussagekräftigere Machbarkeitsstudie zu erstellen. Wir beurteilen diese daher als hilfreich, empfehlen jedoch die in den Konzepten identifizierte, von Feig Electronic produzierte \gls{HF} \gls{RFID} Hardware.

\subsection{\gls{HF} / \gls{UHF}}
Die durch die technische Limitationen kurze Reichweite und grosser Formfaktor der Antennen von \gls{HF} \gls{RFID} verunmöglichte die Entwicklung des Konzeptes zur Suche im Hochregallager. Für die Entwicklung dieses Konzeptes wäre daher \gls{UHF} von Vorteil gewesen. Es hätten sich wahrscheinlich Probleme durch die hohe Dichte von Tags bei \gls{UHF} ergeben. Weiter war die Verwendung von \gls{HF} vom Kunden gewünscht. Für das Konzept Zwei, der Verhinderung der Deplatzierung, erachten wir daher \gls{HF} als die richtige Wahl, für das Konzept Eins, der Suche im Hochregallager, wäre \gls{UHF} vorzuziehen.

\subsection{UnitTests}
Die Verwendung von Unittests in Zusammenarbeit mit der CI/CD-Pipeline führte zu einer besseren Verifikation der Codequalität und Funktionalität. Wir erachten diese daher als hilfreich und den Einsatz deren als erfolgreich.

\subsection{\LaTeX}
Die Verwendung von \LaTeX erlaubte uns die Verwendung des Versionierungssystem git, dadurch den Arbeitsprozess von GitFlow und ständige Reviews, eine CI/CD-Pipeline, und dadurch eine stetige Qualitätssicherung durch das ganze Projekt hindurch. Es erlaubte uns den Fokus auf den Inhalt der Arbeit und nicht auf die Formattierung dieser. Die Eigenheiten von \LaTeX waren uns durch das Vorgängerprojekt bekannt, und verursachten dadurch nur wenige Probleme. Wir erachten den Einsatz von \LaTeX daher als Erfolg für das Projekt.

\subsection{Trello}
Der Einsatz von Trello als digitales SCRUM-Board half uns in der Planung der Sprints und der ganzen Arbeit ungemein. Es ermöglichte beiden Teammitgliedern einen Fokus auf ihre Arbeiten und eine Übersicht auf den Fortschritt des Projektes auf einen Blick. Es war für das Projekt vital und wir verzeichnen es daher als erfolgreich für das Projekt.

\section{Vorgehen}
\subsection{Agile Development}
Der Einsatz der agilen Projektmethode erlaubte uns eine Flexibilität im Projekt, wodurch wir auf Probleme reagieren konnten. Das Problem der Hardwareakquisition konnte daher gelöst werden, ohne dass das Projekt völlig entgleiste. Weiter konnte flexibel auf die Anforderungen des Kunden eingegangen werden, was im Projekt inherent gefordert war, da mehrere Konzepte mit unterschiedlichen Anforderungen und Kapazitäten ausgearbeitet wurden, aber nur eines davon dann für die Machbarkeitsstudie beachtet werden konnte. Wir werten diese daher als einen Erfolg für das Projekt.

\subsection{Sprints und Sprintreviews/Planning}
Die Unterteilung in Zweiwöchige Sprints war rückblickend gesehen die richtige Entscheidung. Eine längere Sprintzeit hätte uns Flexibilität gekostet, während bei einwöchigen Sprints davon auszugehen ist, dass der Überblick auf das Gesamtprojekt verloren gehen würde und generell mehr Stress und mehr indirekte Kosten (im Bezug auf verbrauchte Zeit durch Planung und Review) entstanden wären. Die Reviewsitzungen dauerten durchschnittlich ein bis zwei Stunden und dienten zur Überprüfung der erledigten Arbeiten, Planung des nächsten Vorgehens und zur Teambildung. Die investierte Zeit half zwar nicht produktiv für das Endprodukt, waren aber für den Erfolg des Projektes vital und wird daher als erfolgreich gewertet.

\subsection{Kommunikation}
Durch das ganze Projekt hindurch wurde ein proaktiver Ansatz für die Kommunikation gewählt, sowohl zum Kunden und unserem Betreuer, wie auch teamintern. Verwendet wurde dafür Mail, WhatsApp und Telefon. Nach Rückmeldungen wurde diese offene und zeitnahe Kommunikation vom Kunden sehr geschätzt. Innerhalb des Teams wurde nach einem harzigen Sprint indem die Kommunikation zum erlegen kam, einen wöchentlichen Austausch am Sonntagabend eingeplant. Dieser half erstens, dass man auch über weiche Probleme reden konnte, wie auch dass man über den Fortschritt und die Planung des anderen informiert wurde. Wir schätzen diesen Punkt als extrem wichtig und werten ihn als erfolgreich.

\subsection{Versuche}
Den methodischen Ansatz an die Versuche mit einer vorgängigen Versuchsplanung, Berichterstattung während der Durchführung und anschliessender Auswertung entspricht dem gängigen Vorgehen, war für uns jedoch Neuland. Rückblickend gab es uns Schritt für Schritt vor was zu tun ist und schaffte dadurch Sicherheit und eine höhere Vertrauenswürdigkeit in die gewonnenen Resultate. Wir würden dieses Vorgehen empfehlen und werten es als erfolgreich.

\subsection{Protokolle}
Jedes Meeting mit dem Kunden wurde aufgezeichnet und anschliessend in ein annähernd wörtliches Protokoll niedergeschrieben. Dieses Protokoll half uns auch nach einem Monat wichtige Details noch präsent zu haben und war daher ein wichtiges Hilfsmittel. Einziges Problem dabei war der hohe Arbeitsaufwand der mit der Erstellung des Protokolls verbunden war, eine automatisierte Lösung wäre daher vorzuziehen. Wir schätzen es dennoch als erfolgreich für das Projekt ein.


\chapter{Ausblick}
\label{ch:Ausblick}

Die im Projekt erarbeitete Machbarkeitsstudie ist qualitativ hochwertig entstanden. Sie beschreibt wichtige Erkenntnisse die im Entstehungssprozess gewonnen wurden, und für das Folgeprojekt einen informativen Mehrwert bringt. In diesem Folgeprojekt soll ein Produkt erarbeitet werden, welches sich nahtlos in den Arbeitsprozess der Speicherbibliothek eingliedert.

\section{Projekt Fazit}
Das Projekt gab einen spannenden Einblick in das Gebiet von RFID und deren verschiedenste Arbeitsweise und Techniken. Die angetroffenen Probleme konnten dank der Erfahrung, gewählten Projektweise und dem Einsatz des Projektteams bewältigt werden. Das Projektteam selbst konnte von der vorherigen Zusammenarbeit profitieren und verbesserte die internen Beziehungen und Arbeitsabläufe stetig. Sehr erfreulich war auch die durchgehende ContinousIntegration/ContinousDeplyoment Pipeline die im ganzen Projekt und Projektverlauf eingesetzt wurde, und auch im Projektteam zu einem Continous Improvement führte. Dies vor allem durch eine solide Kommunikation und ausführlichen Sprint-Reviews, aus welchen Lehren gezogen wurden, die für fortführende Sprints eingesetzt wurden.


\newpage

\pagenumbering{Roman}

\appendix

\printglossary

\listoffigures

\listoftables

\listofmyequations \pagebreak

\printbibliography

\chapter{Versuchsdokumentation}
\label{app:ch:versuche}
\chapter{Hardware Spezifikation}
\label{app:ch:hardwarespez}
\includepdf[pages=-]{HYH1WXT_Spec_EN}
\includepdf[pages=-]{HYP3242_Spec_EN}
\chapter{DLL Spezifikationen Hyientech}
\label{app:ch:dllspezifikation}
\includepdf[pages=-]{HyientechDLL_Spec}

\end{document}
