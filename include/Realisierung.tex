\chapter{Realisierung}

\section{Technische Erkenntnisse}

\subsection{Blöcke auf dem Chip}
Durch Auslesen mehrer ausgelehenen Bücher und dem abgleich mit der erhaltenen Datenbank von der Speicherbibliothek, wurde herausgefunden, dass die für die BuchID relevanten Blöcke des RFID Tags die Blöcke 0-3 sind (siehe Tabelle \ref{tbl:ListeBloecke} und Abbildung \ref{fig:AusgeleseneBloeckeUndBarcode}. Weiter werden für die ID des Buches nur schreibbare Unicode-Charaktere verwendet und nicht der geschriebene Hex-Code.

\begin{table}[htb]
	\begin{tabularx}{\textwidth}{|l|l|l|X|}
		\hline
		\textbf{Blocknummer} & \textbf{Inhalt (Hex)} & \textbf{Inhalt (UTF8)} & \textbf{Beschreibung}\\
		\hline
		0 & 11010149 & I & Erster Block der BuchID \\
		\hline
		1 & 4c554d33 & LUM3 & Zweiter Block der BuchID \\
		\hline
		2 & 39303031 & 9001 & Dritter Block der BuchID \\
		\hline
		3 & 31333300 & 133 & Vierter Block der BuchID \\
		\hline
		4 & 00000081 &  & \\
		\hline
		5 & 3e43484c & >CHL & \\
		\hline
		6 & 55485349 & UHSI & \\
		\hline
		7 & 00000000 & - & Leerer Block \\
		\hline
		8 & 00000000 & - & Leerer Block \\
		\hline
		9 & 00000000 & - & Leerer Block \\
		\hline
		10 & 00000000 & - & Leerer Block \\
		\hline
		11 & 00000000 & - & Leerer Block \\
		\hline
		12 & 00000000 & - & Leerer Block \\
		\hline
		13 & 00000000 & - & Leerer Block \\
		\hline
		14 & 00000000 & - & Leerer Block \\
		\hline
		15 & 00000000 & - & Leerer Block \\
		\hline
		16 & 00000000 & - & Leerer Block \\
		\hline
		17 & 00000000 & - & Leerer Block \\
		\hline
		18 & 00000000 & - & Leerer Block \\
		\hline
		19 & 00000000 & - & Leerer Block \\
		\hline
		20 & 00000000 & - & Leerer Block \\
		\hline
		21 & 00000000 & - & Leerer Block \\
		\hline
		22 & 00000000 & - & Leerer Block \\
		\hline
		23 & 00000000 & - & Leerer Block \\
		\hline
		24 & 00000000 & - & Leerer Block \\
		\hline
		25 & 00000000 & - & Leerer Block \\
		\hline
		26 & 00000000 & - & Leerer Block \\
		\hline
		27 & 00000000 & - & Leerer Block \\
		\hline
	\end{tabularx}
	\caption{Blöcke und deren Inhalt für Beispiel HF RFID}
	\label{tbl:ListeBloecke}
\end{table}

\begin{figure}[p]
	\centering
	\begin{subfigure}[t]{.45\textwidth}
		\centering
		\includegraphics[keepaspectratio,width=\linewidth]{RFID_Blocks-Hex}
		\caption{Blöcke in Hex mit Smarphone ausgelesen}
	\end{subfigure}
	\begin{subfigure}[t]{.45\textwidth}
		\centering
		\includegraphics[keepaspectratio,width=\linewidth]{RFID_Blocks-Ascii}
		\caption{Blöcke in ASCII mit Smarphone ausgelesen}
	\end{subfigure}
	\begin{subfigure}[b]{.3\textwidth}
		\centering
		\includegraphics[keepaspectratio,width=\linewidth]{Barcode_BuchRFIDTag}
		\caption{Zugehöriger Barcode}
	\end{subfigure}
	\caption{Ausgelesene Blöcke und Barcode des Buches}
	\label{fig:AusgeleseneBloeckeUndBarcode}
\end{figure}

\section{Systemspezifikation}
\label{sec:SysSpec}

\subsection{Anforderungen}
\label{ch:Anforderungen}

\subsection{Kontext}

\subsection{Komponentendesign}

\subsection{Architektur \& Design}

\subsection{Interne Schnittstellen}

\subsection{Klassendiagramm}

\subsection{Anforderungen der Software}

\subsection{Umsetzung Programmierung}

\subsection{Testing}
