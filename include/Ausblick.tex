\chapter{Ausblick}
\label{ch:Ausblick}

\section{Projekt Fazit}
Das Projekt gab einen spannenden Einblick in das Gebiet von RFID und deren verschiedenste Arbeitsweise und Techniken. Die angetroffenen Probleme konnten dank der Erfahrung, der gewählten Projektweise und dem Einsatz des Projektteams bewältigt werden. Das Projektteam selbst konnte von der vorherigen Zusammenarbeit profitieren. Sehr erfreulich war die durchgehende ContinousIntegration/ContinousDeplyoment Pipeline, welche während des gesamten Projektes und dessen Verlauf eingesetzt werden konnte und damit auch einen starken Beitrag zum Continuous Improvement des Projektteams leistete. Zudem konnte durch eine solide Kommunikation und ausführlichen Sprint-Reviews, aus welchen Lehren gezogen werden konnten, die Zusammenarbeit des Teams in jedem Sprint weiter verbessert werden.

\section{Ausblick}
Die im Projekt entwickelte Machbarkeitsstudie ist qualitativ hochwertig vorhanden und wurde in einer guten und intensiven Teamarbeit entwickelt. Sie beschreibt wichtige Erkenntnisse, die im Entstehungssprozess gewonnen wurden und für das Folgeprojekt einen informativen Mehrwert enthalten. In diesem Folgeprojekt soll ein Produkt erarbeitet werden, welches sich nahtlos in den Arbeitsprozess der Speicherbibliothek eingliedert. Die Erkenntnisse der Machbarkeitsstudie führen zum Schluss, dass das Folgeprojekt trotz technischer Limitationen durchführbar ist und einen finanziellen Mehrwert für den Kunden bringt. Ein Entwicklungsplan zeigt die wichtigsten Meilensteine, die es für das Folgeprojekt zu erreichen gilt.

Bedingt durch unsere Erfahrung würden wir empfehlen, die Hardware von Feig direkt nach Projektstart zu bestellen. Weiter würden wir anraten, dass kurz nach Beginn aktiv der Kontakt zum Hersteller des Lagerverwaltungsystems gesucht wird, damit das Projekt nicht in Verzug gerät. Bezüglich den Tools empfehlen wir \LaTeX zu verwenden. Der Einsatz dieses Schriftsatzsystems bringt mehrere Vorteile. Wir schätzten die einfache Möglichkeit zu zitieren und zu verweisen, sei es ein Buch oder eine Abbildung. Dieser Verweis wird von \LaTeX automatisch korrekt generiert. Ein weiterer Effekt, welcher uns dieses System bot, war die Integration von git mit einer CI/CD Pipeline. Auch diese CI/CD Pipline, zusammen mit dem Feature Branch Workflow Arbeitsprozess, würden wir weiterempfehlen. Dadurch konnte gewährleistet werden, dass keine fehlerhaften Änderungen mit dem Master zusammengeführt wurde. Zusätzlich bietet die Methodik des Feature Branch Workflow auch eine kontinuierliche Überprüfung von Teilstellen der Dokumentation, was normalerweise erst zum Abschluss der Arbeit geschieht. Weiter würden wir bei einer Zweierarbeit wieder zu einem Tool wie Trello raten, da dadurch eine einfache und effiziente Arbeitsaufteilung gewährleistet werden kann.