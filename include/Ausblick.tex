\chapter{Ausblick}
\label{ch:Ausblick}

\section{Projekt Fazit}
Das Projekt gab einen spannenden Einblick in das Gebiet von RFID und deren verschiedenste Arbeitsweise und Techniken. Die angetroffenen Probleme konnten dank der Erfahrung, der gewählten Projektweise und dem Einsatz des Projektteams bewältigt werden. Das Projektteam selbst konnte von der vorherigen Zusammenarbeit profitieren. Sehr erfreulich war die durchgehende ContinousIntegration/ContinousDeplyoment Pipeline, welche während des gesamten Projektes und dessen Verlauf eingesetzt werden konnte und damit auch einen starken Beitrag zum Continuous Improvment des Projektteams leistete. Zudem konnte durch eine solide Kommunikation und ausführlichen Sprint-Reviews, aus welchen Lehren gezogen werden konnten, das Zusammenarbeiten des Teams in jedem Sprint etwas verbessert werden.

\section{Ausblick}
Die im Projekt entwickelte Machbarkeitsstudie ist qualitativ hochwertig vorhanden und wurde in einer guten und intensiven Teamarbeit entwickelt. Sie beschreibt wichtige Erkenntnisse, die im Entstehungssprozess gewonnen wurden, und für das Folgeprojekt einen informativen Mehrwert bringt. In diesem Folgeprojekt soll ein Produkt erarbeitet werden, welches sich nahtlos in den Arbeitsprozess der Speicherbibliothek eingliedert. Die Erkenntnisse der Machbarkeitsstudie führen zum Schluss, dass das Folgeprojekt trotz technischen Limitationen durchführbar ist und einen finanziellen Mehrwert für den Kunden bringt. Ein Entwicklungsplan zeigt die wichtigsten Meilensteine, die es für das Folgeprojekt zu erreichen gilt.

Wie in der Evaluation beschrieben würden wir empfehlen die Dokumentation in \LaTeX zu schreiben, dies da dadurch verschiedene andere Werkzeuge zur Verfügung stehen, die die Qualität der Arbeit verbessern. Projekttechnisch würden wir empfehlen die Meilensteine aus der Machbarkeitsstudie zu übernehmen, die benötigten Tasks zu identifizieren und mit einem Tool wie Trello/ScrumDo zu verfolgen. Dies hilft auch bei einer Einzelarbeit, um den Überblick nicht zu verlieren und nicht in Zeitnot zu geraten.
