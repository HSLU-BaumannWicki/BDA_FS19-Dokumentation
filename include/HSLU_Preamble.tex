%%% Fallback DocumentVersion if Builded local
\providecommand{\docversion}{0.0-localBuild}
\begin{titlepage}
	\begin{textblock*}{5cm}[0,0](15.1cm,1cm)
		\includegraphics[keepaspectratio,width=5cm]{img/HSLU_Logo}
	\end{textblock*}
	\begin{center}
		\vspace*{5cm}
		\Huge{\textbf{Suche von mit RFID ausgerüsteten Einzelexemplaren im vollautomatischen Behälter-Hochregallager}} \\
		\vspace{0.5em}
		\Large{Bachelordiplomarbeit FS2019}\\
		\vspace{3em}
		\LARGE{Pascal Baumann, Dane Wicki}\\
		\vspace{1em}
		\Large{Betreuer: Martin Jud}\\
		\vfill
		\large{Hochschule Luzern - Departement Informatik}\\
		\large{\today}\\
		\large{Version \docversion}
	\end{center}
	\begin{textblock*}{5cm}[0,0](15.3cm,277mm)
		\includegraphics[keepaspectratio,width=5cm]{img/FHZ_Logo}
	\end{textblock*}
\end{titlepage}

\newpage

\pagenumbering{gobble}

\begin{textblock*}{5cm}[0,0](15cm,0.7cm)
	\includegraphics[keepaspectratio,width=2.7cm]{img/HSLU_Logo_Header}
\end{textblock*}

\vspace*{1.35cm}

\noindent
\textbf{\Large{Bachelorarbeit an der Hochschule Luzern - Informatik}}

\vspace{0.6cm}
\noindent
\textbf{Titel:} RFID markierte Exemplare

\vspace{0.6cm}
\noindent
\textbf{Student 1:} Pascal Baumann

\vspace{0.6cm}
\noindent
\textbf{Student 2:} Dane Wicki

\vspace{1cm}
\noindent
\textbf{Studiengang:} BSc Informatik

\vspace{0.6cm}
\noindent
\textbf{Abschlussjahr:} 2019

\vspace{0.6cm}
\noindent
\textbf{Betreuungsperson:} Martin Jud

\vspace{0.6cm}
\noindent
\textbf{Experte:} Urs Gehrig

\vspace{0.6cm}
\noindent
\textbf{Codierung / Klassifizierung der Arbeit:}

\begin{todolist}
	\item \textbf{A: Einsicht (Normalfall)}
	\item \textbf{B: Rücksprache}\hspace*{0.7cm}(Dauer:\hspace*{1cm} Jahr / Jahre)
	\item \textbf{C: Sperre}\hspace*{1.865cm}(Dauer:\hspace*{1cm} Jahr / Jahre)
\end{todolist}

\vfill

\noindent
\textbf{Eidesstattliche Erklärung}
\\
Ich erkläre hiermit, dass ich/wir die vorliegende Arbeit selbständig und ohne unerlaubte fremde Hilfe angefertigt haben, alle verwendeten Quellen, Literatur und andere Hilfsmittel angegeben haben, wörtlich oder inhaltlich entnommene Stellen als solche kenntlich gemacht haben, das Vertraulichkeitsinteresse des Auftraggebers wahren und die Urheberrechtsbestimmungen der Fachhochschule Zentralschweiz (siehe Markblatt «Studentische Arbeiten» auf MyCampus) respektieren werden.

\vspace{1em}

\noindent
\begin{tabularx}{\textwidth}{@{}lX}
	&\\
	Ort / Datum, Unterschrift: &  \\
	\cline{2-2}
	&\\[0.5cm]
	Ort / Datum, Unterschrift: &  \\
	\cline{2-2}
\end{tabularx}

\begin{textblock*}{5cm}[0,0](14.93cm,277mm)
	\includegraphics[keepaspectratio,width=5cm]{img/FHZ_Logo}
\end{textblock*}

\newpage

\begin{textblock*}{5cm}[0,0](15cm,0.7cm)
	\includegraphics[keepaspectratio,width=2.7cm]{img/HSLU_Logo_Header}
\end{textblock*}

\noindent
\textbf{Abgabe der Arbeit auf der Portfolio Datenbank}

\vspace{0.5em}

\noindent
\textbf{Bestätigungsvisum Studentin / Student}
\\
\noindent
Ich bestätige, dass ich die Bachelorarbeit korrekt gemäss Merkblatt auf der Portfolio Datenbank abgelegt habe. Die Verantwortlichkeit sowie die Berechtigungen habe ich abgegeben, so dass ich keine Änderungen mehr vornehmen kann oder weitere Dateien hochladen kann.

\vspace{0.7em}

\noindent
\begin{tabularx}{\textwidth}{@{}lX}
	&\\
	Ort / Datum, Unterschrift: &  \\
	\cline{2-2}
	&\\[0.5cm]
	Ort / Datum, Unterschrift: &  \\
	\cline{2-2}
\end{tabularx}

\vspace{0.8cm}
\noindent
\textbf{Verdankung}
\\
An dieser Stelle möchten wir uns bei all denjenigen bedanken, die uns bei der Erstellung dieser Bachelorarbeit unterstützt haben. Als Erstes möchten wir uns bei unserem Betreuer Martin Jud bedanken, der unsere Bachelorarbeit unterstützt und begutachtet hat. Für die hilfreichen Empfehlungen zur Struktur dieser Arbeit und den Hilfestellungen zur Anschaffung der Versuchshardware möchten wir uns herzlich bedanken. Weiter möchten wir uns bei Urs Gehrig für seinen Einsatz als Experten für unsere Arbeit bedanken. Einen Dank auch an Herrn Mike Märki, welcher diese Arbeit überhaupt erst ermöglicht hat, und uns erlaubte unsere Referenzimplementation vor Ort zu testen. Ganz herzlich möchten wir uns auch bei Jeremy Meier bedanken, welcher uns in der Architektur beraten und ein Review der Systemspezifikation geliefert hat.

\vspace{0.8cm}
\noindent
\textbf{Eingangsvisum (durch das Sekretariat auszufüllen):}

\noindent
\renewcommand{\arraystretch}{2}
\begin{tabularx}{\textwidth}{@{}lXlX}
	Rotkreuz, den & & Visum: & \\
	\cline{2-2}
	\cline{4-4}
\end{tabularx}
\renewcommand{\arraystretch}{1}

% Uncomment for print_version

% \vfill
% \noindent
% \textbf{Hinweis}: Die Bachelorarbeit wurde von keinem Dozierenden nachbearbeitet. Veröffentlichungen (auch auszugsweise) sind ohne das Einverständnis der Studiengangleitung der Hochschule Luzern – Informatik nicht erlaubt.

% \vspace{1em}
% 
% \noindent
% \textbf{Copyright} © 2019 Hochschule Luzern - Informatik
% 
% \vspace{1em}
% \noindent
% Alle Rechte vorbehalten. Kein Teil dieser Arbeit darf ohne die schriftliche Genehmigung der Studiengangleitung der Hochschule Luzern - Informatik in irgendeiner Form reproduziert oder in eine von Maschinen verwendete Sprache übertragen werden.

\begin{textblock*}{5cm}[0,0](14.93cm,277mm)
	\includegraphics[keepaspectratio,width=5cm]{img/FHZ_Logo}
\end{textblock*}
