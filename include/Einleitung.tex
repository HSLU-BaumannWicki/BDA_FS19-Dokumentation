\chapter{Einleitung}
\label{ch:Einleitung}
In der Kooperativen Speicherbibliothek Schweiz werden Exemplare der dem Verein zugehörigen Bibliotheken eingelagert. Das heisst, das Exemplar der Bibliothek wird gereinigt und inventarisiert eingelagert. Sollte ein Bibliotheksbenutzer ein Exemplar dieses Werks anfordern, wird das Exemplar per Kurier an die jeweilige Bibliothek verschickt. Journale oder Magazine werden vorzugsweise eingescannt und in digitaler Form an den Benutzer übergeben, sofern dies so vom Benutzer bestellt wurde. Sowohl die beteiligten Bibliotheken, wie auch die Bibliotheksbenutzer sind daher an einer zeitgerechten Lieferung interessiert. Dies bedeutet für die Speicherbibliothek, dass alle Prozesse und Abläufe effizient und zuverlässig ablaufen müssen. Verzögerungen in Zwischenschritten können sich auf die ganze Auslieferung des Exemplars, in physischer oder digitaler Form, auswirken.

Im Hochregallager der Speicherbibliothek werden momentan bis zu 110'000 Behältern mit verschiedenen Exemplaren, von welchen viele mit RFID Tags ausgestattet sind, gelagert. Die Behälter werden manuell von Menschen befüllt und entleert, und anschliessend voll autonom an einen Lagerplatz gefahren. Zeitweise können auch gewisse Exemplare wieder aus den Behältern entnommen werden, um dieses Exemplar zu Lesen, Scannen, der teilnehmenden Bibliotheken zurückzusenden, oder aus dem Lagerbestand zu entfernen. Der Vorgang des Lagern und Entnehmens birgt die Gefahr, dass eine Person aus Versehen ein Exemplar in einen falschen Behälter einsortiert. Dies führt zu dem Umstand, dass das Exemplar nur sehr umständlich wiedergefunden werden kann \parencite{WickiBaumann2019Projektbeschrieb}.

In dieser Arbeit wird daher eine Lösung für das Problem der Deplatzierung als Konzept ausgearbeitet, dieses Konzept in einer Machbarkeitsstudie geprüft und mit einer Referenzimplementation validiert.

\section{Aufgabenstellung und Zielsetzung}
Ziel dieser Arbeit ist es, mittels einer Machbarkeitsstudie und einem Proof of Concept zu untersuchen, ob es technisch realisierbar ist, Exemplare, welche mit RFID Tags ausgerüstet sind, im Hochregallager der Speicherbibliothek zu finden \parencite{WickiBaumann2019Projektbeschrieb}.

Daraus leiteten sich die folgenden Artefakte ab, welche dem Kunden abgegeben werden:
\begin{itemize}
	\item Zwei ausgearbeiteten Konzepte
	\item Eine Machbarkeitsstudie zu einem ausgewählten Konzept
	\item Ein Versuchsprotokoll, basiernd auf den vor Ort durchgeführten Versuchen
\end{itemize}
