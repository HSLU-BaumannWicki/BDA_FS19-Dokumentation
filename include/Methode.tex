\chapter{Methode}

\section{Projektinformationen}

\subsection{Vorgehensmodell}

Alle Wirtschaftsprojekte an der Hochschule Luzern fallen in eine der folgenden Kategorien:

\begin{enumerate}
	\item Einsatz von Standardsoftware und Services
	\item Software- und Produktentwicklung
	\item Innovationsprojekt
	\item IT-Infrastrukturentwicklung
	\item Strukturierte Analyse und Konzeption von Systemen und Abläufen
\end{enumerate}

Dabei ist dieses Projekt als Innovationsprojekt und Softwareentwicklung klassifiziert worden. Wir erwarteten daher unter anderem, eine Evaluation, Recherchen und weitere Unbekannten. Um auf diese eingehen zu können, entschied sich das Team dafür die hybride, inkrementelle Agile Methode zu verwenden.

\subsection{Agile Projektmethode}

Die agile Projektmethode zielt darauf ab in einem ungewissen und sich verändernden Umfeld zu bestehen. Insbesondere bedeutet dies, das auf sich verändernde Voraussetzungen schnell reagiert werden kann und dabei ein funktionierendes Produkt entsteht \parencite{AgileAlliance2015}. Dies soll durch eine enge Zusammenarbeit mit dem Auftraggeber und guter teaminterner Kommunikation erreicht werden.

\parencite{BaumannWicki2018}

\subsection{Ermittlung offener Projektrahmenbedingungen}
\label{ch:evaluation}

\subsection{Projektanforderungen}
Mittels einer Machbarkeitsstudie und einem Proof of Concept soll untersucht werden ob es möglich ist bis zu 120 \gls{rfid} Tags in einem Behälter mit der Dimension 600x400x320mm zu identifizieren.

\begin{itemize}
	\item Es sollen mindestens zwei Lösungskonzepte für eine als Auswahl der Machbarkeitsstudie entwickelt werden.
	\item Die Lösungskonzepte müssen auf deren technische Realisierbarkeit untersucht werden.
	\item Es muss mindestens ein entwickeltes Konzept für die Machbarkeitsstudie verwendet werden.
	\item Die Machbarkeitsstudie muss eine Kostenrechnung für die Lösungsansätze beinhalten.
	\item Es soll eine MVP entwickelt werden, welches vom Kunde verwendet werden kann.
\end{itemize}

\subsubsection{Anforderungen an Lösungsansätze, Proof of Concept und MVP}

\begin{itemize}
	\item Die Lösungskonzepte müssen mit dem Lagersystem kommunizieren können
	\item Die Lösungskonzepte müssen die \gls{rfid} Tags in weniger als 1 Sekunden identifizieren können.
	\item Die Lösungskonzepte müssen für das bestehende Hochregallager der Speicherbibliothek verwendbar sein.
	
	\item Das Proof of Concept muss technisch aufzeigen, wie viele \gls{rfid} Tags in einer Sekunde gelesen werden können.
	\item Das Proof of Concept soll eine \gls{rfid} Lesezuverlässigkeit von 95\% aufweisen.
	
	\item Das MVP soll mit der Datenbank des Lagersystems kommunizieren können.
	\item Das MVP soll in einem von Störfaktoren bereinigten Zustand die gleiche Anzahl \gls{rfid} Tags lesen können wie im Proof of Concept definiert.
	\item Das MVP soll erkennen, wenn eine Box ein Exemplar enthält, welches nicht dieser Box zugehörig ist und dies als eine Unstimmigkeit markieren.
	\item Das MVP soll in der Lage sein, dem Endbenutzer in beliebiger Form mitzuteilen, welcher Behälter eine Unstimmigkeit enthält.
\end{itemize}

\subsection{Einschränkungen und Abgrenzungen}

\section{Machbarkeitsstudie}
