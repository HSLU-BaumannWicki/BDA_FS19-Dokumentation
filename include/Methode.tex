\chapter{Methode}

\section{Projektinformationen}

\subsection{Vorgehensmodell}

Alle Wirtschaftsprojekte an der Hochschule Luzern fallen in eine der folgenden Kategorien:

\begin{enumerate}
	\item Einsatz von Standardsoftware und Services
	\item Software- und Produktentwicklung
	\item Innovationsprojekt
	\item IT-Infrastrukturentwicklung
	\item Strukturierte Analyse und Konzeption von Systemen und Abläufen
\end{enumerate}

Dabei ist dieses Projekt als Innovationsprojekt und Softwareentwicklung klassifiziert worden. Wir erwarteten daher unter anderem, eine Evaluation, Recherchen und weitere Unbekannten. Um auf diese eingehen zu können, entschied sich das Team dafür die hybride, inkrementelle Agile Methode zu verwenden.

\subsection{Agile Projektmethode}

Die agile Projektmethode zielt darauf ab in einem ungewissen und sich verändernden Umfeld zu bestehen. Insbesondere bedeutet dies, das auf sich verändernde Voraussetzungen schnell reagiert werden kann und dabei ein funktionierendes Produkt entsteht \parencite{AgileAlliance2015}. Dies soll durch eine enge Zusammenarbeit mit dem Auftraggeber und guter teaminterner Kommunikation erreicht werden.

\parencite{BaumannWicki2018}

\subsection{Ermittlung offener Projektrahmenbedingungen}
\label{ch:evaluation}
\subsection{Projektanforderungen}
\label{sub:Anforderungen}
Mittels einer Machbarkeitsstudie und einem Proof of Concept soll untersucht werden ob es möglich ist bis zu 120 \gls{RFID} Tags in einem Behälter mit der Dimension 600x400x320mm zu identifizieren.
Um dieses Ziel zu erreichen wurden verschiedenen Anforderungen ermittelt.

\begin{legal}
	\item Es sollen mindestens zwei Lösungskonzepte für eine als Auswahl der Machbarkeitsstudie entwickelt werden.
	\item Die Lösungskonzepte müssen auf deren technische Realisierbarkeit untersucht werden.
	\item Es muss mindestens ein entwickeltes Konzept für die Machbarkeitsstudie verwendet werden.
	\item Die Machbarkeitsstudie muss eine Kostenrechnung für die Lösungsansätze beinhalten.
	\item Es soll eine Referenzimplementation entwickelt werden, welche vom Kunde verwendet werden kann.
	
	\item Anforderungen an Konzepte
	\begin{legal}
		\item Die Lösungskonzepte müssen mit dem Lagersystem kommunizieren können
		\item Die Lösungskonzepte müssen die RFID Tags in weniger als 1 Sekunden identifizieren können.
		\item Die Lösungskonzepte müssen für das bestehende Hochregallager der Speicherbibliothek verwendbar sein.
	\end{legal}
	\item Anforderungen an Proof of Concept
	\begin{legal}
		\item Das Proof of Concept muss technisch aufzeigen, wie viele RFID Tags in einer Sekunde identifiziert werden können.
	\end{legal}
	\item Anforderungen an Referenzimplementation
	\begin{legal}
		\item Die Referenzimplementation ist in der Lage die Buch ID eines Exemplares über RFID auszulesen.
		\item Die Referenzimplementation soll erkennen, wenn eine Box ein Exemplar (eines, welches mit RFID ausgestattet ist und technisch auch Lesbar ist) enthält, welches nicht dieser Box zugehörig ist.
		\item Die Referenzimplementation soll jede erkannte Unstimmigkeit (Exemplar, welches nicht zu diesem Behälter gehört) in einem Logdokument persistieren.
		\item Die Referenzimplementation soll in der Lage sein, dem Endbenutzer, in graphischer Form durch eine Konsolen-Ausgabe, mitzuteilen, welcher Behälter eine Unstimmigkeit enthält.
		\item Die Referenzimplementation soll die unter Laborbedingungen erhaltenen Resultate unter Realbedingungen verifizieren.
		\item Die Referenzimplementation soll mit einer Oracle Datenbank kommunizieren können.
	\end{legal}
\end{legal}

\subsection{Einschränkungen und Abgrenzungen}
Um die Konzepte sowie Versuche vor Ort zu validieren, wurde eine Referenzimplementation entwickelt. Diese zu einer Produktionsreife zu bringen, war explizit nicht Teil des Projektes. Zur Entwicklung der Referenzimplementation wurde ein Datenbankauszug verwendet und nicht eine Schnittstelle zur real verfügbaren Datenbank. Die Entwicklung einer produktionsreifen Applikation wurde in der Planung des Folgeprojektes, im Rahmen der Machbarkeitsstudie, berücksichtigt. Auf die Form dieser Applikation wurde jedoch nicht genauer eingegangen, sodass diese Spezifikation im Ermessen des nächsten Projektteames liegt.

Einschränkungen ergaben sich durch das beschränkte Budget, wodurch auf einen alternativen Hersteller ausgewichen werden musste, welcher eine kleinere Reichweite als gewünscht lieferte. Die Resultate können daher auch nur explizit für diese spezifische Hardware gewährleistet werden.
Weiter bestand Teamintern ein kleiner Wissensstand über die Durchführung einer Machbarkeitsstudie zu Beginn des Projektes, welcher im Laufe des Projektes zuerst erarbeitet werden musste.

\section{Machbarkeitsstudie}
Die Machbarkeitsstudie diente dazu ein ausgewähltes Konzept zu validieren und auf deren Machbarkeit zu untersuchen. Insbesondere wurde auf die technische und wirtschaftliche Machbarkeit wert gelegt. Da die Erarbeitung einer Machbarkeitsstudie eine Neuheit für beide Teammitglieder war, wurde zuerst in einer Recherchenphase offene Fragen geklärt. Dabei wurde eine Anleitung zur Erarbeitung einer Machbarkeitsstudie erstellt (siehe Kapitel \ref{app:ch:AnleitungMachbarkeitsstudie}). Diese stützt sich in grossen Teilen auf eine Anleitung des US Departement of Agriculture \parencite{Matson2000}.
