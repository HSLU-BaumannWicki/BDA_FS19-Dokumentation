\chapter{Evaluation und Validation}
\label{ch:Eval}
In diesem Kapitel wird evaluiert, ob das Projekt und dessen produzierte Artefakte die Anforderungen erfüllen. Dazu wurden alle Anforderungen nochmals zusammengetragen und jede einzelne mit dem jeweiligen Artefakt überprüft, das Resultat wurde schliesslich in Form einer einfach lesbaren Tabelle festgehalten. Aus dem Vergleich der konnte schliesslich festgestellt werden, dass alle priorisierten Anforderungen erfüllt werden konnten. Daher kann dieses Projekt als erfolgreich betrachtet werden.

\section{Vergleich mit Anforderungen}
\label{sec:VergleichAnforderungen}
Hier werden die Anforderungen aufgelistet und ob diese erfüllt wurden oder nicht. Sofern diese nicht erfüllt wurden, wird für deren Erfüllung eine potenzielle Lösungsidee beschrieben. Für die Verifikation wurden die Testfälle überprüft.

Legende:
\begin{itemize}[label={}, noitemsep]
	\item <<\checkmark >> = Anforderung erfüllt
	\item <<\xmark>> = Anforderung nicht erfüllt
	\item <<\notmark>> Wurde aus den beschriebenen Gründen nicht Implementiert.
\end{itemize}


\begin{tabularx}{\textwidth}{l l X}
	\hline
	\checkmark & 1   & Es sollen mindestens zwei Lösungskonzepte für eine als Auswahl der Machbarkeitsstudie entwickelt werden. \\
	\hline
	\checkmark & 2   & Die Lösungskonzepte müssen auf deren technische Realisierbarkeit untersucht werden. \\
	\hline
	\checkmark & 3   & Es muss mindestens ein entwickeltes Konzept für die Machbarkeitsstudie verwendet werden. \\
	\hline
	\checkmark & 4   & Die Machbarkeitsstudie muss eine Kostenrechnung für die Lösungsansätze beinhalten. \\
	\hline
	\checkmark & 5   & Es soll eine Referenzimplementation entwickelt werden, welche vom Kunde verwendet werden kann. \\
	\hline
	\checkmark & 6.1 & Die Lösungskonzepte müssen mit dem Lagersystem kommunizieren können \\
	\hline
	\checkmark & 6.2 & Die Lösungskonzepte müssen die RFID Tags in weniger als 1 Sekunden identifizieren können. \\
	\hline
	\checkmark & 6.3 & Die Lösungskonzepte müssen für das bestehende Hochregallager der Speicherbibliothek verwendbar sein. \\
	\hline
	\checkmark & 7.1 & Das Proof of Concept muss technisch aufzeigen, wie viele RFID Tags in einer Sekunde identifiziert werden können. \\
	\hline
	\checkmark & 8.1 & Die Referenzimplementation ist in der Lage die Buch ID eines Exemplares über RFID auszulesen. \\
	\hline
	\checkmark & 8.2 & Die Referenzimplementation soll erkennen, wenn eine Box ein Exemplar (eines, welches mit RFID ausgestattet ist und technisch auch Lesbar ist) enthält, welches nicht dieser Box zugehörig ist. \\
	\hline
	\checkmark & 8.3 & Die Referenzimplementation soll jede erkannte Unstimmigkeit (Exemplar, welches nicht zu diesem Behälter gehört) in einem Logdokument persistieren. \\
	\hline
	\checkmark & 8.4 & Die Referenzimplementation soll in der Lage sein, dem Endbenutzer, in graphischer Form, durch eine Konsolen-Ausgabe, mitzuteilen, welcher Behälter eine Unstimmigkeit enthält. \\
	\hline
	? & 8.5 & Die Referenzimplementation soll die unter Laborbedingungen erhaltenen Resultate unter Realbedingungen verifizieren. \\
	\hline
	\notmark & 8.6 & Die Referenzimplementation soll mit einer Oracle Datenbank kommunizieren können. \\
	\hline
\end{tabularx}

\subsection{Begründung des Nichtumsetzens von 8.6}
Während der Implementationsphase konnte festgestellt werden, dass die Einbindung der Oracle Datenbank mit grösserem Zeitaufwand verbunden ist, als dem Team noch zur Verfügung stand. Der Umstand, dass diese Anforderung in der Meilensteinsitzung 3 mit einer geringen Priorität versehen wurde (siehe Anhang \ref{app:sec:protokollMeilenstein3}), führte dazu, dass diese nicht umgesetzt wurde.

\section{Technische Aspekte}
\subsection{ContinuousIntegration / ContinuousDeployment}
Im Projekt wurde durchgehend Travis für CI/CD verwendet. Dies bedeutete, dass sowohl die Dokumentation, wie auch die erstellten Applikationen, bei jedem Push gebaut wurden. Beim Build auf einem Masterbranch wurde dieser auch gleich als einen Prerelease auf GitHub deployed.

Fehlerhafte Änderungen in einem Push konnten mit dieser Pipeline schnell erkannt und beseitigt werden. Zudem bestand jederzeit die Möglichkeit, auf eine ältere Version zurückzugreifen, da jede Änderung auf dem Master Branch veröffentlicht wird. Bedingt durch die gewonnenen Möglichkeiten und  dem geringen Einrichtungsaufwand bringt dieses Hilfsmittel einen sehr grossen Mehrwert und wird daher als äusserst effektiv beurteilt.

\subsection{HF RFID Hardware}
Die evaluierte \gls{HF} \gls{RFID} Hardware von Hyintech erreichte nicht die gewünschte Reichweite und funktionierte nur auf einem 32bit Windows System (aufgrund der gelieferten \gls{DLL}). Sie ermöglichte es dennoch, unsere Versuche durchzuführen und eine aussagekräftigere Machbarkeitsstudie zu erstellen. Da sie aber durch die limitierte Reichweite einen Behälter nicht komplett durchdringt und dieser dadurch nicht komplett gescannt werden kann, raten wir von der Verwendung für die Produktivrealisierung ab. Wir empfehlen daher die in den Konzepten identifizierte, von Feig Electronic produzierte \gls{HF} \gls{RFID} Hardware.

\subsection{HF / UHF}
Die durch die technischen Limitationen kurze Reichweite und grosser Formfaktor der Antennen von \gls{HF} \gls{RFID} verunmöglichte die Entwicklung des Konzeptes zur Suche im Hochregallager. Für die Entwicklung dieses Konzeptes würde daher \gls{UHF} als Alternative in Frage kommen, zumal es auch schon Projekte im Bibliotheksumfeld mit \gls{UHF} gibt. Wie in der Recherche eruiert, bietet \gls{UHF} eine höhere Reichweite, schnellere Datenraten und preiswertere Hardware. Es hätten sich wahrscheinlich Probleme durch die hohe Dichte von Tags bei \gls{UHF} ergeben. Weiter war die Verwendung von \gls{HF} vom Kunden gewünscht. Für das Konzept Zwei, der Verhinderung des Deplatzierens, erachten wir daher \gls{HF} als die richtige Wahl, für das Konzept Eins, der Suche im Hochregallager, wäre \gls{UHF} vorzuziehen.

\subsection{UnitTests}
Die Verwendung von Unittests, in Zusammenarbeit mit der CI/CD-Pipeline, führte zu einer besseren Verifikation der Codequalität und Funktionalität. Zudem konnten dank den vorhanden Tests Refactorings durchhgeführt werden, ohne das Vertrauen in die Applikation zu verlieren. Diese erwähnten Vorteile machten es möglich, dass auf Anpassungen vor Ort einfacher reagiert werden konnte. Durch diesen Mehrwert erachten wir den Einsatz von UnitTest als sehr erfolgreich.

\subsection{\LaTeX}
Die Verwendung von \LaTeX erlaubte uns die Verwendung des Versionierungssystem git. Dadurch konnte mit Feature Branches gearbeitet werden, bei welchen durch die nötigen Pull Request, ständigen Reviews durch das andere Teammitglied folgten. Weiter konnte eine CI/CD Pipline eingerichtet werden, welche dafür sorgte, dass auf Build Fehler schnell reagiert werden und fehlerhafte \LaTeX-Dateien ausfindig gemacht werden konnten. Weiter erlaubte es uns, den Fokus auf den Inhalt der Arbeit zu richten und nicht auf die Formatierung dieser. Der Einsatz von \LaTeX war uns durch das Vorgängerprojekt bereits bekannt und verursachte dadurch nur geringe Probleme. Durch diese wenigen Probleme, wie auch die Integration von git und die damit verbundenen positiven Aspekte veranlassen uns dazu, den Einsatz von \LaTeX als erfolgreich für dieses Projekt zu betrachten.

\subsection{Trello}
Der Einsatz von Trello als digitales SCRUM-Board half uns in der Planung der Sprints und der ganzen Arbeit. Es ermöglichte beiden Teammitgliedern einen Fokus auf ihre Arbeiten und eine Übersicht auf den Fortschritt des Projektes auf einen Blick. Die Möglichkeit die Schätzung von Story Points aufgrund von Fakten zu verbessern hat uns bei diesem Tool gefehlt. Für kommende Projekte ist es anzuraten, sofern die Zeiten von Soll und Ist miteinander verglichen werden sollen, ein anderes Tool einzusetzen, oder dieses durch ein weiteres zu ergänzen. Wir betrachten daher den Einsatz von Trello als Teilerfolg.

\section{Vorgehen}
\subsection{Agile Development}
Der Einsatz der agilen Projektmethode erlaubte uns eine Flexibilität im Projekt, wodurch wir auf Probleme reagieren konnten. Das Problem der Hardwareakquisition konnte daher gelöst werden, ohne dass das Projekt völlig entgleiste. Weiter konnte flexibel auf die Anforderungen des Kunden eingegangen werden, was im Projekt inhärent gefordert war, da mehrere Konzepte mit unterschiedlichen Anforderungen und Kapazitäten ausgearbeitet wurden, aber nur eines davon für die Machbarkeitsstudie beachtet werden konnte. Wir werten diese daher als einen Erfolg für das Projekt.

\subsection{Sprints und Sprintreviews/Planning}
Die Unterteilung in Zweiwöchige Sprints war rückblickend gesehen die richtige Entscheidung. Eine längere Sprintzeit hätte uns Flexibilität gekostet, während bei einwöchigen Sprints davon auszugehen ist, generell mehr Stress und mehr indirekte Kosten (im Bezug auf verbrauchte Zeit durch Planung und Review) entstanden wären. Die Reviewsitzungen dauerten durchschnittlich ein bis zwei Stunden und dienten zur Überprüfung der erledigten Arbeiten, Planung des nächsten Vorgehens und zur Teambildung. Die investierte Zeit half zwar nicht produktiv für das Endprodukt, waren aber für den Erfolg des Projektes vital und wird daher als erfolgreich gewertet.

\subsection{Kommunikation}
Durch das gesamte Projekt hindurch wurde ein proaktiver Ansatz für die Kommunikation gewählt, sowohl zum Kunden und unserem Betreuer, wie auch teamintern. Verwendet wurde dafür Mail, WhatsApp und Telefon. Nach Rückmeldungen wurde diese offene und zeitnahe Kommunikation vom Kunden sehr geschätzt. Innerhalb des Teams wurde nach einem nicht nach Plan verlaufenden Sprint, indem die Kommunikation zum Erliegen kam, einen wöchentlichen Austausch am Sonntagabend eingeplant. Dieser Austausch führte dazu, dass auch marginale Probleme besprochen werden konnten, sowie der Sprintfortschrit und somit den Fortschritt der Teammitglieder überprüft werden konnte. Die Kommunikation zum Kunden betrachten wir, unter Betrachtung des positiven Feedbacks, als erfolgreich. Die Teaminterne Kommunikation ist nach der Definition des Austauschtermins deutlich verbessert worden, weshalb diese Kommunikation als erfolgreich erachtet werden kann.

\subsection{Versuche}
Den methodischen Ansatz an die Versuche mit einer vorgängigen Versuchsplanung, Berichterstattung während der Durchführung und anschliessender Auswertung entspricht dem gängigen Vorgehen, war für uns jedoch Neuland. Rückblickend gab es uns Schritt für Schritt vor, was zu tun ist und schaffte dadurch Sicherheit und eine höhere Vertrauenswürdigkeit in die gewonnenen Resultate. Durch die erhaltene Resultate, welche mit dem methodischen Ansatz gute Schlussfolgerungen zulassen, erachten wir die Durchführung der Versuche als erfolgreich.

\subsection{Protokolle}
Jedes Meeting mit dem Kunden wurde aufgezeichnet und anschliessend in ein annähernd wörtliches Protokoll niedergeschrieben. Dieses Protokoll half uns wichtige Details jederzeit zu überprüfen und war daher ein wichtiges Hilfsmittel. Einziger Nachteil dabei war der hohe Arbeitsaufwand, der mit der Erstellung des Protokolls verbunden war, eine automatisierte Lösung wäre daher vorzuziehen. Wir schätzen es dennoch als Teilerfolg für das Projekt ein.
