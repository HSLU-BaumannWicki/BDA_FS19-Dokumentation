\chapter{Evaluation und Validation}
\label{ch:Eval}
In diesem Kapitel wird evaluiert, ob das Projekt und dessen produzierte Artefakte die Anforderungen erfüllen. Dazu wurden alle Anforderungen nochmals zusammengetragen und jede einzelne mit dem jeweiligen Artefakt überprüft, das Resultat wurde schliesslich in Form einer einfach lesbaren Tabelle festgehalten. Aus dem Vergleich der konnte schliesslich festgestellt werden, dass alle priorisierten Anforderungen erfüllt werden konnten. Daher kann dieses Projekt als erfolgreich betrachtet werden.

\section{Vergleich mit Anforderungen}
\label{sec:VergleichAnforderungen}
Hier werden die Anforderungen aufgelistet und ob diese erfüllt wurden oder nicht. Sofern diese nicht erfüllt wurden, wird für deren Erfüllung eine potenzielle Lösungsidee beschrieben. Für die Verifikation wurden die Testfälle überprüft.

Legende:
\begin{itemize}[label={}, noitemsep]
	\item <<\checkmark >> = Anforderung erfüllt
	\item <<\xmark>> = Anforderung nicht erfüllt
	\item <<\notmark>> Wurde aus den beschriebenen Gründen nicht Implementiert.
\end{itemize}


\begin{tabularx}{\textwidth}{l l X}
	\hline
	\checkmark & 1   & Es sollen mindestens zwei Lösungskonzepte für eine als Auswahl der Machbarkeitsstudie entwickelt werden. \\
	\hline
	\checkmark & 2   & Die Lösungskonzepte müssen auf deren technische Realisierbarkeit untersucht werden. \\
	\hline
	\checkmark & 3   & Es muss mindestens ein entwickeltes Konzept für die Machbarkeitsstudie verwendet werden. \\
	\hline
	\checkmark & 4   & Die Machbarkeitsstudie muss eine Kostenrechnung für die Lösungsansätze beinhalten. \\
	\hline
	\checkmark & 5   & Es soll eine Referenzimplementation entwickelt werden, welche vom Kunde verwendet werden kann. \\
	\hline
	\checkmark & 6.1 & Die Lösungskonzepte müssen mit dem Lagersystem kommunizieren können \\
	\hline
	\checkmark & 6.2 & Die Lösungskonzepte müssen die RFID Tags in weniger als 1 Sekunden identifizieren können. \\
	\hline
	\checkmark & 6.3 & Die Lösungskonzepte müssen für das bestehende Hochregallager der Speicherbibliothek verwendbar sein. \\
	\hline
	\checkmark & 7.1 & Das Proof of Concept muss technisch aufzeigen, wie viele RFID Tags in einer Sekunde identifiziert werden können. \\
	\hline
	\checkmark & 8.1 & Die Referenzimplementation ist in der Lage die Buch ID eines Exemplares über RFID auszulesen. \\
	\hline
	\checkmark & 8.2 & Die Referenzimplementation soll erkennen, wenn eine Box ein Exemplar (eines, welches mit RFID ausgestattet ist und technisch auch Lesbar ist) enthält, welches nicht dieser Box zugehörig ist. \\
	\hline
	\checkmark & 8.3 & Die Referenzimplementation soll jede erkannte Unstimmigkeit (Exemplar, welches nicht zu diesem Behälter gehört) in einem Logdokument persistieren. \\
	\hline
	\checkmark & 8.4 & Die Referenzimplementation soll in der Lage sein, dem Endbenutzer, in graphischer Form, durch eine Konsolen-Ausgabe, mitzuteilen, welcher Behälter eine Unstimmigkeit enthält. \\
	\hline
	? & 8.5 & Die Referenzimplementation soll die unter Laborbedingungen erhaltenen Resultate unter Realbedingungen verifizieren. \\
	\hline
	\notmark & 8.6 & Die Referenzimplementation soll mit einer Oracle Datenbank kommunizieren können. \\
	\hline
\end{tabularx}

\subsection{Begründung des Nichtumsetzens von 8.6}
Während der Implementationsphase konnte festgestellt werden, dass die Einbindung der Oracle Datenbank mit grösserem Zeitaufwand verbunden ist, als dem Team noch zur Verfügung stand. Der Umstand, dass diese Anforderung in der Meilensteinsitzung 3 mit einer geringen Priorität versehen wurde (siehe Anhang \ref{app:sec:protokollMeilenstein3}), führte dazu, dass diese nicht umgesetzt wurde.

\section{Technische Aspekte}
\subsection{ContinuousIntegration / ContinuousDeployment}
Im Projekt wurde durchgehend Travis für CI/CD verwendet, dies bedeutete, dass sowohl die Dokumentation wie auch die erstellten Applikationen bei jedem Push gebaut wurden, und auf dem Masterbranch direkt als Prerelease deployed wurde.

Es konnten so Fehler direkt erkannt und behoben werden, und es bestand die Möglichkeit auf alte Versionen direkt zuzugreifen. Wir beurteilen dieses Hilfsmittel als extrem effektiv.

\subsection{\gls{HF} \gls{RFID} Hardware}
Die evaluierte \gls{HF} \gls{RFID} Hardware der Hyintech erreichte nicht die gewünschte Reichweite und funktionierte nur auf einem 32bit Windows System (aufgrund der gelieferten DLL). Sie ermöglichte es dennoch, unsere Versuche durchzuführen und eine aussagekräftigere Machbarkeitsstudie zu erstellen. Wir beurteilen diese daher als hilfreich, empfehlen jedoch die in den Konzepten identifizierte, von Feig Electronic produzierte \gls{HF} \gls{RFID} Hardware.

\subsection{\gls{HF} / \gls{UHF}}
Die durch die technische Limitationen kurze Reichweite und grosser Formfaktor der Antennen von \gls{HF} \gls{RFID} verunmöglichte die Entwicklung des Konzeptes zur Suche im Hochregallager. Für die Entwicklung dieses Konzeptes wäre daher \gls{UHF} von Vorteil gewesen. Es hätten sich wahrscheinlich Probleme durch die hohe Dichte von Tags bei \gls{UHF} ergeben. Weiter war die Verwendung von \gls{HF} vom Kunden gewünscht. Für das Konzept Zwei, der Verhinderung der Deplatzierung, erachten wir daher \gls{HF} als die richtige Wahl, für das Konzept Eins, der Suche im Hochregallager, wäre \gls{UHF} vorzuziehen.

\subsection{UnitTests}
Die Verwendung von Unittests in Zusammenarbeit mit der CI/CD-Pipeline führte zu einer besseren Verifikation der Codequalität und Funktionalität. Wir erachten diese daher als hilfreich und den Einsatz deren als erfolgreich.

\subsection{\LaTeX}
Die Verwendung von \LaTeX erlaubte uns die Verwendung des Versionierungssystem git, dadurch den Arbeitsprozess von GitFlow und ständige Reviews, eine CI/CD-Pipeline, und dadurch eine stetige Qualitätssicherung durch das ganze Projekt hindurch. Es erlaubte uns den Fokus auf den Inhalt der Arbeit und nicht auf die Formattierung dieser. Die Eigenheiten von \LaTeX waren uns durch das Vorgängerprojekt bekannt, und verursachten dadurch nur wenige Probleme. Wir erachten den Einsatz von \LaTeX daher als Erfolg für das Projekt.

\subsection{Trello}
Der Einsatz von Trello als digitales SCRUM-Board half uns in der Planung der Sprints und der ganzen Arbeit ungemein. Es ermöglichte beiden Teammitgliedern einen Fokus auf ihre Arbeiten und eine Übersicht auf den Fortschritt des Projektes auf einen Blick. Es war für das Projekt vital und wir verzeichnen es daher als erfolgreich für das Projekt.

\section{Vorgehen}
\subsection{Agile Development}
Der Einsatz der agilen Projektmethode erlaubte uns eine Flexibilität im Projekt, wodurch wir auf Probleme reagieren konnten. Das Problem der Hardwareakquisition konnte daher gelöst werden, ohne dass das Projekt völlig entgleiste. Weiter konnte flexibel auf die Anforderungen des Kunden eingegangen werden, was im Projekt inherent gefordert war, da mehrere Konzepte mit unterschiedlichen Anforderungen und Kapazitäten ausgearbeitet wurden, aber nur eines davon dann für die Machbarkeitsstudie beachtet werden konnte. Wir werten diese daher als einen Erfolg für das Projekt.

\subsection{Sprints und Sprintreviews/Planning}
Die Unterteilung in Zweiwöchige Sprints war rückblickend gesehen die richtige Entscheidung. Eine längere Sprintzeit hätte uns Flexibilität gekostet, während bei einwöchigen Sprints davon auszugehen ist, dass der Überblick auf das Gesamtprojekt verloren gehen würde und generell mehr Stress und mehr indirekte Kosten (im Bezug auf verbrauchte Zeit durch Planung und Review) entstanden wären. Die Reviewsitzungen dauerten durchschnittlich ein bis zwei Stunden und dienten zur Überprüfung der erledigten Arbeiten, Planung des nächsten Vorgehens und zur Teambildung. Die investierte Zeit half zwar nicht produktiv für das Endprodukt, waren aber für den Erfolg des Projektes vital und wird daher als erfolgreich gewertet.

\subsection{Kommunikation}
Durch das ganze Projekt hindurch wurde ein proaktiver Ansatz für die Kommunikation gewählt, sowohl zum Kunden und unserem Betreuer, wie auch teamintern. Verwendet wurde dafür Mail, WhatsApp und Telefon. Nach Rückmeldungen wurde diese offene und zeitnahe Kommunikation vom Kunden sehr geschätzt. Innerhalb des Teams wurde nach einem harzigen Sprint indem die Kommunikation zum erlegen kam, einen wöchentlichen Austausch am Sonntagabend eingeplant. Dieser half erstens, dass man auch über weiche Probleme reden konnte, wie auch dass man über den Fortschritt und die Planung des anderen informiert wurde. Wir schätzen diesen Punkt als extrem wichtig und werten ihn als erfolgreich.

\subsection{Versuche}
Den methodischen Ansatz an die Versuche mit einer vorgängigen Versuchsplanung, Berichterstattung während der Durchführung und anschliessender Auswertung entspricht dem gängigen Vorgehen, war für uns jedoch Neuland. Rückblickend gab es uns Schritt für Schritt vor was zu tun ist und schaffte dadurch Sicherheit und eine höhere Vertrauenswürdigkeit in die gewonnenen Resultate. Wir würden dieses Vorgehen empfehlen und werten es als erfolgreich.

\subsection{Protokolle}
Jedes Meeting mit dem Kunden wurde aufgezeichnet und anschliessend in ein annähernd wörtliches Protokoll niedergeschrieben. Dieses Protokoll half uns auch nach einem Monat wichtige Details noch präsent zu haben und war daher ein wichtiges Hilfsmittel. Einziges Problem dabei war der hohe Arbeitsaufwand der mit der Erstellung des Protokolls verbunden war, eine automatisierte Lösung wäre daher vorzuziehen. Wir schätzen es dennoch als erfolgreich für das Projekt ein.
